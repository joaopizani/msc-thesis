\chapter{Conclusions}
\label{chap:conclusions}

    Hardware design is an important and \emph{complex} activity, with strict requirements.
    Furthermore, its relevance is growing with more need for hardware acceleration,
    pushing programmers and other professionals to think about hardware without having
    the traditional engineering background.

    Functional programming has for a long time allowed for programming in a more
    abstract and principled fashion, farther from details of any specific machine.
    Also, very frequently, building a system using a purely functional programming language
    makes that system easier to analyze and easier for its properties to be stated and proven.

    Given these advantages when developing software, it should come as no surprise that
    a long-standing line of research is concerned with applying the ideas of functional
    programming to hardware design.
    Functional hardware description languages, and later
    \emph{embedded} domain-specific languages for hardware were developed,
    and several studies have shown that some of the advantages brought to software development
    by adopting the functional paradigm could also benefit the world of hardware.

    More recently, programming languages with \emph{dependent type systems} were developed,
    creating the paradigm of \acf{DTP}.
    Many researchers regard \ac{DTP} as being the "next step up" in terms of abstraction,
    when compared to functional programming.
    Nothing more reasonable then than to investigate whether \emph{even more} benefits
    can be brought to the field of hardware design by \ac{DTP},
    improving upon the advances of functional hardware description languages.

    Inspired by the ideas in Coquet~\cite{coquet2011}, and by the strenghts and weaknesses of other
    hardware \acp{DSL} studied in a previous project, we developed in this thesis
    a hardware \ac{DSL} – Π-Ware – \emph{embedded in a dependently-typed language} and using
    as much as possible of the cutting-edge developments in \ac{DTP}.
    The development of Π-Ware had an exploratory character: having identified limitations of
    other functional hardware \acp{DSL}, we tried to use features of dependent types in general
    (and of Agda in specific) to solve them.

    Π-Ware does allow for more type safety in circuit designs,
    prevents some classes of frequent design mistakes and provides a framework
    in which both simulation, synthesis and verification of hardware can be performed \emph{in the same language}.
    We use dependent inductive families to ensure basic \emph{well-formedness} of circuits,
    dependent records (type classes) and recursive instance search to allow for parametrization
    in the types handled by circuits and the underlying circuit technology used (gate family),
    and the Curry-Howard isomorphism~\cite{propositions-as-types} gives us a logic interpretation of Agda,
    in which we can write \emph{types} stating properties of circuits and give \emph{proofs} for them.

    However, there are still some "rough edges": features that due to low priority and/or time constraints
    are currently not as polished as we wish them to be.
    Furthermore, during the development of Π-Ware, some interesting ideas came up about the application
    of \ac{DTP} to hardware development which deserve future investigation.


    \section{Future work}
    \label{sec:future-work}

        \subsection{Current limitations and trade-offs in Π-Ware}
        \label{subsec:current-limitations}

            At the beginning of Π-Ware's development, we were considering
            how should the low-level circuit type (\AD{ℂ'}) be indexed.
            It was clear that the two indices should characterize in some way the circuit's \emph{interface},
            that is, its input and output ports.
            Initially, we defined a \emph{closed universe} of types which low-level circuit descriptions
            could "operate on". This universe was, essentially, closed under products (\AD{\_×\_}),
            coproducts (\AD{\_×\_}) and vectors.
            Had this attempt gone forward, the type of a low-level description for \AF{⊻ℂ'} (\texttt{XOR})
            might have looked as follows – where \AI{A} indicates one \AD{Atom}:

            \begin{center}
                \AF{⊻ℂ'} : \AD{ℂ'} \AY{(}\AI{A} \AI{⊠} \AI{A}\AY{)} \AI{A}
            \end{center}

            However, trying to enforce the desired well-formedness rules and perform simulation
            with this universe construction turned out to be more complex than expected,
            and ultimately we decided to use \emph{sizes} as the indices of \AD{ℂ'}.
            This decision mainly impacts the synthesis semantics:
            with no \emph{structural} information about circuit ports in the low-level,
            the generated \ac{VHDL} has a \mintinline{vhdl}{Port} declaration with only
            \emph{one} input and \emph{one} output (of appropriate size).

            Another current trade-off in Π-Ware concerns the \ARR{⇓W⇑} (\texttt{Synthesizable})
            type class: high-level simulation semantics (\AF{⟦\_⟧}) \emph{requires} that the
            \AL{⇓} (serialization) and \AL{⇑} (deserialization) functions be \emph{inverses} of each other.
            Currently, however, this requirement is \emph{not} enforced by requiring a \emph{proof} of
            this fact as a field of \ARR{⇓W⇑}.

            The argument for not enforcing this requirement through the type system is to allow
            prototyping of circuits without the burden of proof for all types involved.
            On the other hand, the current implementation is certainly not the only way to provide this flexibility:
            We could require the proof of inversibility as a field of \ARR{⇓W⇑} anyways,
            and expect the designer to \AK{postulate} the proofs while prototyping.
            When ready for synthesis, the \texttt{safe} flag can be used with Agda to forbid all \AK{postulates}.

            Lastly, describing and trying to prove facts about complex circuit definitions in Π-Ware
            can be quite inefficient: type-checking a generically-sized ripple-carry adder can already
            consume up to 2GB of RAM.
            This is not, however, an issue exclusively with Π-Ware; other Agda libraries suffer with the same problem.
            Agda's own documentation \emph{wiki} suggests, in a page dedicated to perfomance tuning,
            to use some of Agda's experimental features, such as:

            \begin{itemize}
                \item Proof \emph{irrelevance}, when a proof argument is not used in the body of a definition.
                \item Marking proofs as \AK{abstract}, so that they are not unfolded.
            \end{itemize}

            \newpage

        \subsection{Directions for future research}
        \label{subsec:future-research}
            As mentioned briefly in Section~\ref{subsec:synthesizable}, any finite Agda datatype can have
            an instance of \ARR{⇓W⇑} (\texttt{Synthesizable}), given a certain encoding.
            One further research path would, therefore, involve \emph{generic programming};
            more specifically, creating a sums-of-products view of Agda datatypes – similar to
            what has been done in the paper "True Sums of Products"~\cite{true-sop}.

            It would also be necessary to define a generic one-to-one correspondence
            between the elements of these sums-of-products and vectors of \AD{Atom}s,
            independently of what \ARR{Atomic} instance is used.
            Haskell's \texttt{binary}\footnote{\url{https://hackage.haskell.org/package/binary}}
            library is likely to provide some inspiration in this regard.
            Both of these endeavours would involve using Agda's
            \emph{reflection}~\cite{engineering-reflection-agda} mechanism:
            in order to create a sum-of-products encoding for any (finite) datatype,
            and to extract a serialized sequence of \AD{Atom}s from the sum-of-products.

            A generally desirable effort would be to prove \emph{in Agda} several
            algebraic and meta-theoretical properties of Π-Ware.
            For example, properties such as associativity, commutativity and identity of
            circuit constructors and combinators could be proven, allowing them to fit into
            the structures defined in the \AM{Algebra} module of Agda's standard library.
            Some properties could even help to guide possible optimization steps in the synthesis:
            proving μFP's~\cite{mufp-1984} "state law" regarding the \AI{DelayLoop} constructor,
            for example, could enable big memory blocks to be subdivided and placed nearer
            to the combinational block operating on their contents.

            Lastly, Π-Ware adopts a relatively \emph{low-level} approach to system description:
            we describe gates, sizes, wires, etc. It would certainly be interesting to investigate
            the usage of \acl{DTP} for modelling other levels of abstraction, such as
            instruction sets, processor microarchitectures, etc. in a \emph{layered} fashion.
