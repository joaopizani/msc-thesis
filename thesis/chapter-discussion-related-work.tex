\chapter{Discussion and related work}
\label{chap:discussion-related-work}

    The goals of Π-Ware and the techniques used in the library were very much inspired by the
    \acp{EDSL} already mentioned in Sections~\ref{subsec:embedded-functional-hardware}
    and~\ref{sec:hardware-dtp}, some of which were more deeply studied in a project
    predating this thesis.

    In this chapter we provide an overviwe of Π-Ware's features by means of example circuits and
    drawing comparisons with related work.

    \section{Functional hardware \acp{EDSL}}
    \label{sec:discussion-functional}

        \subsubsection{μFP}
        Π-Ware is a \emph{deep}-embedded \ac{DSL}, and was made so in order to allow for multiple
        interpretations or \emph{semantics} of circuits.
        The mapping between Π-Ware circuits and different semantics is similar to the behaviour
        of μFP~\cite{mufp-1984}.
        Each constructor of Π-Ware's low-level circuit datatype (\AD{ℂ'}) corresponds to two
        semantics, as in μFP: the \emph{simulation} semantics and the \emph{synthesis} semantics.
        Even though μFP is not embedded, and therefore does not have "circuit constructors" as
        (deep) \acp{EDSL} would have, we can still relate μFP's composition to Π-Ware's sequencing,
        μFP's tupling to Π-Ware parallel composition and so forth, keeping the analogy.

        There is a fundamental difference in what regards the \emph{synthesis} semantics of Π-Ware
        and μFP: the synthesis semantics of μFP prescribes a \emph{floorplan}, while Π-Ware synthesis
        only specifies a \emph{netlist}.
        A \emph{netlist} is a directed graph with logic gates as nodes and signals (wires) as edges.
        Producing a \emph{floorplan} goes one (slight) step further, by specifying relative
        physical positions between the components, such as "besides", "adjacent to", etc.
        In Π-Ware's case, it is the job of \emph{other} tools in the implementation chain to take
        Π-Ware's generated \ac{VHDL} as input and perform placement and routing.

        A last idea from μFP which inspired Π-Ware greatly was the definition of a
        single operator capable of introducing \emph{state} into circuits.
        It is also the only way to introduce loops in circuits.
        This construct is called the "μ combining form" in μFP, and actually gave the language its name,
        as μFP is an extension of Backus's FP~\cite{backus-turing-lecture} with exactly this one
        extra combining form.

        The Π-Ware equivalent of μ is called the \AI{DelayLoop} constructor, but it works in exactly the same way:
        given a combinational circuit, it "loops" part of the output back into that circuit's input,
        passing it through a memory element (a \emph{clocked latch}).
        Being the \emph{only} way to introduce loops, it also ensures that circuits contain no
        \emph{combinational loops}.

        The example of a shift register – with its μFP geometrical semantics shown in
        Figure~\ref{fig:mufp-shift} – has the following equivalent description in Π-Ware:

        \begin{center}
            \ExecuteMetaData[code/agda/latex/PiWare/Samples/BoolTrioSeq.tex]{shift}
        \end{center}

        When simulated, this circuit produces an output stream in which the tail is a copy of the
        input stream and the head is an "undefined" \AD{Atom} (the \AD{Atom} indexed by \AI{zero}).
        The output stream can be said to be "shifted" by one clock cycle, thus the circuit's name.

        Making use of dependent types, Π-Ware "tags" circuits with information on whether or not
        they have cycles (and thus state).
        In this way, \emph{combinational simulation} (which disregards past inputs) only happens
        when a circuits is \emph{provably} cycle-free.

        \subsubsection{Lava}
        From Lava~\cite{observable-sharing-circuits}, Π-Ware borrowed the idea of \emph{connection patterns}.
        Connection patterns are functions that, given one or more circuits as arguments,
        produce a larger, more powerful circuit by connecting its arguments in a certain \emph{pattern}.
        Usually these connection patterns are parameterized by a \emph{size},
        and are defined by induction on it.

        One example in Lava of such a connection pattern is \mintinline{haskell}{compose},
        the sequential composition of a list of circuits.
        The code for \mintinline{haskell}{compose} is shown on Listing~\ref{lst:lava-compose}.

        \begin{listing}[h]
            \haskellfile{code/haskell/LavaCompose.hs}
            \caption{Lava's \mintinline{haskell}{compose} connection pattern.\label{lst:lava-compose}}
        \end{listing}

        And for comparison, Π-Ware's version of it (named \AF{seqs'}) is shown in Listing~\ref{lst:seqs-core}.

        \begin{listing}[h]
            \centering{\ExecuteMetaData[code/agda/latex/PiWare/Patterns/Core.tex]{seqs-core}}
            \caption{Π-Ware's version of the \texttt{compose} connection pattern.\label{lst:seqs-core}}
        \end{listing}

        Even though \AF{seqs'} is written by using a fold on the vector of circuits and
        Lava's \mintinline{haskell}{compose} uses explicit recursion, the behaviour of both is the same:
        they connect the given circuit in series, i.e.,
        with the output of circuit $n$ connected to the input of circuit $n+1$.
        It is important to notice, however, how Π-Ware uses an \emph{explicit} sequential combination
        \emph{constructor} in the fold (\AI{\_⟫'\_}), while Lava uses Haskell's own function application syntax for
        this – the output of the first circuit in the list (\mintinline{haskell}{x = circ inp})
        is fed to the remainder of the "chain".

        In Lava's \mintinline{haskell}{compose}, an ordinary Haskell \texttt{List} is used
        as the collection of circuits to be connected.
        Therefore, as Haskell's lists are homogeneous, all of its elements – in this case,
        all of the circuits to be combined – must have the \emph{same type}.
        Using Agda, this unnecessary constraint imposed upon the collection of circuits could
        be lifted, and a more general pattern achieved.

        The minimum requirement to connect a collection of circuits in series is that their \emph{interfaces match},
        i.e., the output of circuit $n$ has the same size as the input of circuit $n+1$.
        The \AM{PiWare.Patterns} module is still very much a work-in-progress,
        and therefore such a generalized sequencing pattern is not yet implemented,
        but we expect that the concept of \emph{typed lists} as defined in a paper by
        McKinna and Wright~\cite{typed-stack-safe-compiler} will be useful to capture this notion of
        \emph{constrained} heterogeneous list.

        Moving away from connection patters, one aspect in which Π-Ware improves significantly
        upon Lava is the \emph{size checking} of circuits.
        As already mentioned briefly in Section~\ref{subsec:embedded-functional-hardware},
        Lava's usage of Haskell lists to express generically-sized circuit definitions leaves some
        mistakes to be detected only at runtime.

        For example, in the case of a generically-sized adder, the two inputs are expected to be
        of the \emph{same size}, but the \texttt{List} type cannot provide this guarantee,
        thus if mismatching inputs are provided, the error will only be detected at runtime.
        Notice how in the example Lava definition of a binary adder,
        the case with input lists of mismatching sizes is left undefined,
        making a pattern match failure possible:

        \begin{center}
            \haskellfile{code/haskell/LavaAdder.hs}
        \end{center}

        In contrast, Π-Ware makes use of dependent types and has its circuit type (\AD{ℂ'})
        \emph{indexed by the size} of circuit inputs and outputs, thus enforcing these constraints easily.
        Listing~\ref{lst:ripple} shows the example of a ripple-carry added modelled in Π-Ware.

        \begin{listing}[h]
            \centering{\ExecuteMetaData[code/agda/latex/PiWare/Samples/RippleCarry.tex]{ripple}}
            \caption{Example of a ripple-carry adder modelled in Π-Ware.\label{lst:ripple}}
        \end{listing}

        Notice how the input to the circuit defined in \AF{ripple} is a product where two of the factors
        \emph{must be of the same type}, namely, a boolean vector of size $n$, for any $n$.
        Furthermore, the description \emph{style} changed drastically: in Π-Ware, circuits are described
        in a \emph{nameless} fashion, i.e., internal signals cannot be named and connected arbitrarily,
        \emph{plugs} must be used to do the job of \emph{rewiring} (reordering, forking, etc.).
        Some examples of plugs used in \AF{ripple} are \AF{pCons}, \AF{pALR} and \AF{pIntertwine},
        all of them defined in \AM{PiWare.Plugs} together with several others.

        Finally, perhaps the biggest improvement of Π-Ware when compared to Lava and similar \acp{EDSL}
        is the possibility to use high-level, Agda types to model and simulate circuits.
        In contrast, Lava's circuit descriptions are essentially untyped – there is only a choice
        between \mintinline{haskell}{Bool}, \mintinline{haskell}{Int}, and tuples and lists thereof.
        For example, when modelling a CPU one might want to have a datatype definition corresponding
        to instructions of that CPU. In Lava, the best bet would be a simple \emph{type synonym},
        something along these lines:

        \begin{haskellcode}
        type SB = Signal Bool
        type ALUControlBits = (SB, SB, SB, SB, SB, SB)
        type HackInstruction = (SB, SB, SB, SB, SB, SB, SB, SB, SB, SB, ALUControlBits)
        \end{haskellcode}

        This description is not only cumbersome to work with
        (tuples don't have as many useful associated functions as sized vectors),
        but also error-prone.
        Any other tuple of 16 bits which might happen to be in scope
        (for example, from a register bank or some other part of the circuit),
        can be passed \emph{by mistake} to a function accepting a \mintinline{haskell}{HackInstruction},
        and the type checker will not be able to detect this mistake.

        In Π-Ware, one can normally design a custom datatype for \AD{HackInstruction} and, by defining
        an instance of \AD{⇓W⇑} (\texttt{Synthesizable}) for it, have circuits operating over this
        high-level datatype.

        \subsubsection{ForSyDe}

        ForSyDe~\cite{forsyde1999} and Π-Ware are both deeply embedded, but in ForSyDe the user
        writes combinational circuits as Haskell functions and then they are \emph{reified}
        by Template Haskell into an \ac{AST} that ForSyDe actually uses to build the circuit at compile-time.
        On the other hand, circuits in Π-Ware are written with a specific syntax, using specific constructors
        and combinators.

        In terms of type safety, ForSyDe already improved a little over Lava, by essentially
        \emph{emulating dependent types}.
        While ForSyDe's "emulation" is far from the full power of dependent types
        (implementing only type-level naturals and statically-sized vectors),
        it is already enough to make circuit descriptions more "typed", less error prone and
        easier to understand.

        The range of types which can be \emph{synthesized} with ForSyDe, however, is still small
        compared to Π-Ware. The already mentioned datatype for a CPU instruction, for example, can
        easily be used as a circuit's input or output type in Π-Ware, but not in ForSyDe.
        In ForSyDe, the most typed possible representation would have an \emph{enumerated type} for the
        \emph{opcodes}~\footnote{Shortening of "operation codes",
            usually written in assembly language as \texttt{ADD}, \texttt{LOAD}, etc.},
        but simple sized vectors for the other fields (addresses, etc.);
        ForSyDe can synthesize enumerations, but not other, more complex forms of user-defined datatypes.

        This limitation has to do with the different ways in which Π-Ware and ForSyDe define
        the semantics of a \emph{synthesizable type}.
        In ForSyDe, to synthesize a datatype means to actually translate it to an equivalent in \ac{VHDL}.
        Standard \ac{VHDL}~\cite{vhdl-standard} has (limited) typing capabilities
        (namely, enumerations and records are supported), so ForSyDe will translate
        a Haskell \mintinline{haskell}{Int16} to a correspondingly ranged integer in VHDL,
        and a Haskell \mintinline{haskell}{OpCode} enumerated type will result in a corresponding
        \ac{VHDL} enumeration.

        Π-Ware, on the other hand, generates untyped \ac{VHDL}, where all circuit ports and signals
        are vectors of the chosen \AD{Atom} type (for example, \mintinline{vhdl}{std_logic}).
        In this way, any type which can be mapped into a vector of \AD{Atom}s – by writing an
        instance of \AD{⇓W⇑} – can be used as circuit IO, and the circuit will be able to be synthesized.
        This design decision was taken as we believe that all correctness checking and architectural
        manipulation of circuits should be done at the meta (Agda) level anyway,
        so when a \ac{VHDL} netlist is generated, a typed representation would not bring enough benefits.
        Furthermore, the typed \ac{VHDL} generated by ForSyDe is problematic in terms of tool support.
        When – in the experimentation project~\cite{functional-hardware-survey} carried out as preparation
        for this thesis – we tried to implement ForSyDe-generated \ac{VHDL} on an \ac{FPGA},
        it could be processed by one manufacturer's toolchain (Altera™) but not the other (Xilinx™).

        To conclude, when comparing Π-Ware to functional hardware \acp{EDSL} in general,
        the most significant feature is the capability to \emph{prove properties of circuit behaviour},
        and to be able to formulate statements and proofs
        \emph{in the same language as the circuits themselves}.
        Some functional hardware \acp{EDSL} (Lava, for example), are capable of exporting formulas
        for model checking with \emph{external} tools (such as SAT solvers), but only by using
        dependently-typed programming proofs can be written in the same language as circuits are described.

        %% TODO: example proof here? being too repetitive?


    \section{Dependently-Typed Hardware \acp{EDSL}}

        \draftnote{Note to reviewer: talk about Brady's correct-by-construction adder?~\cite{brady-constructing}}

        With Coquet~\cite{coquet2011} Π-Ware shares the basic ideas about circuit \emph{syntax}.
        Both use a \emph{strucutural} representation, in which specific constructors of the circuit
        datatype represent patterns of connection between subcircuits.
        Also, both Π-Ware and Coquet use indexed data families to represent circuits, in which
        the indices carry information about the circuit's input and output ports.

        The difference lies in exactly what indices are used: Π-Ware's
        circuits~\footnote{low-level circuits, in the \AD{ℂ'} family}
        are indexed simply by two natural numbers, representing the \emph{size} of a circuit's input and output.
        For example, the type of an n-bit adder in Π-Ware would be something like the following:

        \begin{center}
            \AF{adder} \AY{:} \AD{ℂ'} \AY{(}\AB{n} \AF{+} \AB{n}\AY{)} \AY{(}\AI{suc} \AB{n}\AY{)}
        \end{center}

        An instance of such an adder (for a particular $n$) can add two n-bit numbers, producing
        an output of size $n+1$. The information that this circuit has \emph{two} inputs, however,
        is not expressed in the signature – The signature says that the circuit has
        \emph{one input of size $n+n$}.
        Π-Ware's low-level descriptions (which are the ones effectively synthesized) are translated
        into \ac{VHDL} entities with \emph{always} one input and one output.
        The \emph{knowledge of how the circuit's ports are organized} is not present in the \ac{VHDL}
        level, only in Agda.

        Coquet, on the other hand, allows for more \emph{structured} types to be used as indices in
        the \mintinline{coq}{Circuit} data family.
        In fact, any type belonging to Coquet's \mintinline{coq}{Fin} (finite) class can be used.
        The problem, however, is that the requirement imposed upon types in order to belong to
        \mintinline{coq}{Fin} is that they need to be \emph{enumerable}, that is, there needs to be a
        list defined with all the type's elements.
        Enumerations, products, coproducts, etc. all can easily have instances of this type class.
        But the class provides \emph{no information on how to synthesize them}, i.e.,
        how to write the \ac{VHDL} \emph{port declaration} when a circuit's input is a complex
        sum-of-products.

