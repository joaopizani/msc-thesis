\begin{abstract}
    Recently, the incentives for hardware acceleration of algorithms are growing,
    as Moore's law continues to hold but optimizations in traditional processors
    show diminishing returns. This growing need for hardware implementation pushes
    programmers towards hardware design, an activity with stricter correctness and
    performance requirements than software development.
    A long-standing line of research is concerned with the application of functional
    programming techniques to hardware design, and the relatively recent
    dependently-typed programming paradigm has been claimed by many researchers to be
    the "successor" of functional programming.
    This thesis aims, therefore, to investigate what are the improvements that
    Dependently-Typed Programming (DTP) can bring to hardware design.
    We developed a domain-specific language for hardware, called Π-Ware,
    embedded as a library in the Agda programming language.
    Π-Ware allows the specification of circuits, their simulation, synthesis and verification.
    Compared to similar approaches in the literature, Π-Ware provides
    high type safety and robust proofs of correctness for whole families of circuits,
    however it still needs significant improvements in terms of ease-of-use.
\end{abstract}
