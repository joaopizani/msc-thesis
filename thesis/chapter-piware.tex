\chapter{The Π-Ware library}
\label{chap:piware}
    Lorem ipsum\ldots

    %% How to present:
        %% By design decisions ∨

    \section{Circuit Syntax}
    \label{sec:circuit-types}
        \begin{itemize}
            \item Shallow vs. deep embedding
                \subitem Chose deep because of synthesis and non-functional analyses
            \item Structural modelling
                \subitem Represent connections among circuits in the DSL, \emph{not in the metalanguage}
                \subitem Avoid having to deal with observable sharing
        \end{itemize}

    \section{Abstraction Mechanisms}
    \label{sec:abstraction}
        \begin{itemize}
            \item Data abstraction (atom vectors x arbitrary synthesizable types)
                \subitem Atom (enumeration axioms)
                \subitem Word / Synthesizable
                    \subsubitem Proof search?
            \item Gate abstraction (Technology primitives x Semantic fundamentals)
                \subitem Circuits descriptions are parameterized by a set of fundamental gates
                    \subsubitem Always combinational
                    \subsubitem Their correctness is assumed
                \subitem For synthesis, still need to give these gates a description in terms of technology primitives
        \end{itemize}

    \section{Circuit Semantics}
    \label{sec:eval-seq}
        \begin{itemize}
            \item Two types of evaluation: combinational and sequential
            \item Combinational eval requires a proof that the circuit contains no loops
                \subitem Eval of a fundamental gate is just its \emph{definitional behaviour}

            \item For sequential circuits we use a \emph{causal stream semantics}
            %% TODO: couldn't we restrict it even more and just keep ONE PAST VALUE in the context?
                \subitem Current output depends on the current input and (possibly) on the past input
                \subitem \emph{Different} than plain Stream functions

            \item There's also an eval function which allows the circuit to be viewed as function from Stream to Stream
        \end{itemize}
